\chapter{论文的内容要求}
\label{chap:contents}
\par {\bSong 一份完整的毕业设计(论文)包括:标题、基本信息、承诺书、摘要、关键词、目录、正文、参考文献、致谢、附录等。}

\section{标题}
\par 标题包括中文标题和外文标题。

\subsection{中文标题}
\par 中文标题应该简短、明确、有概括性,不宜超过20个汉字。

\subsection{外文标题}
\par 外文标题应该简短、明确、翻译正确。

\section{基本信息}
\par 基本信息包括:学院名称、专业名称、作者姓名、作者学号、指导教师姓名及职称,以及论文完成的日期。

\section{承诺书}
\par 承诺书是论文作者对学术诚信的庄重承诺。本规范提供了上海理工大学本科毕业设计(论文)承诺书的一个规范文本,作者在认真仔细阅读后签上姓名和日期。

\section{摘要}
\par 摘要包括中文摘要和外文摘要两部分。
\par 中外文摘要均包括正文和关键词。

\subsection{摘要正文}
\par 论文摘要简要陈述本科毕业设计(论文)的内容,创新见解和主要论点。中文摘要在500字左右,外文摘要应与中文摘要的内容相符。

\subsection{关键词}
\par 关键词是反映毕业设计(论文)主题内容的名词,是供检索使用的。关键词条应为通用词汇,不得自造关键词。关键词一般为3至5个,按其外延层次,由高至低顺序排列。关键词排在摘要正文部分下方。

\section{目录}
\par 目录按三级标题编写,要求标题层次清晰,并标明页码。

\section{正文}
\par 正文篇幅要求15000字以上(其中,英语、德语专业毕业设计(论文)应不少于5000外文单词,日语专业应不少于8000日语假名,艺术设计类专业不少于3000字。毕业设计(论文)的核心设计、研究篇幅应占全篇幅的三分之二以上。)。内容包括绪论、正文主体与结论。

\subsection{绪论}
\par 绪论是研究工作的概述,内容包括:本课题的意义、目的、研究范围及要达到的技术要求;简述本课题在国内外的发展概况及存在的问题。
\par 绪论一般作为在毕业设计(论文)正文的第1章,并在一章内完成。

\subsection{正文主体}
\par 正文主体是研究工作的详述,内容包括:问题的提出,研究工作的基本前提、假设和条件;模型的建立,实验方案的拟定;基本概念和理论基础;设计计算的主要方法和内容;实验方法、内容及其分析;理论论证及其应用,研究结果,以及对结果的讨论等。
\par 正文主体一般可分为若干章完成。

\subsection{结论}
\par 结论是研究工作的总结,内容包括:对所得结果与已有结果的比较和课题尚存在的问题,以及进一步展开研究的见解与建议。
\par 结论一般作为论文正文的最后一章,并在一章内完成。

\section{参考文献}
\par 参考文献为研究中参考的资料,包括专著、论文、年鉴、网站等。所引用的文献必须是公开发表的与毕业设计(论文)工作直接有关的文献,且经过本人阅读理解。列入的主要文献要求不少于6篇,其中外文文献不少于2篇。

\section{致谢}
\par 毕业设计(论文)是在指导教师的指导下完成,理应致谢。还应对完成论文提供过帮助的其他人员致谢。切忌泛滥和溢美。

\section{附录}
\par 附录是对于一些不宜放在正文中,但又直接反映研究工作的材料(如设计图纸、实验数据、计算机程序等)附于文本末尾。
