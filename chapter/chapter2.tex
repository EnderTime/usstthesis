\chapter{论文的格式要求}
\label{chap:formats}
\par 论文的格式要求包括:纸张大小、纸张方向、页边距、版式、文档网格、字体与字号、段落和间距等\cite{机器学习}。
\par 建议采用 \LaTeX2e 语法和 \verb|xelatex| 引擎编译。
\par 由于论文格式问题非常繁杂,无法将所有设置描述清楚,只能对一些主要的设置做出扼要的说明。一个快捷有效的方法就是把本规范的电子版作为模板。
\par 小贴士:使用 \verb|xelatex| 编译时,可以直接生成双面打印的 PDF 文件。如果需要编译两次生成交叉引用,第一次编译时加入 \verb|--no-pdf| 选项可以加快编译速度。

\section{页面设置}
\subsection{纸张}
\par 纸张大小:A4。
\par 纸张方向:纵向。

\subsection{页边距}
\par 页边距;上2.5厘米,下2.5厘米,内侧3厘米,外侧2.5厘米。
\par 页码范围:对称页边距。

\subsection{版式}
\par 节:奇数页。
\par 页眉和页脚:奇偶页不同,距边界:页眉1.5厘米,页脚1.75厘米。

\subsection{文档网格}
\par 网格:无网格。

\subsection{字体}
\par 中文字体:宋体,使用{\bSong 宋体粗体}时,请使用 \verb|{\bSong 要加粗的文字}|。
\par 西文字体:Times New Roman。
\par 字形:常规。
\par 字号:小四。

\subsection{段落}
\par 对齐方式:两端对齐。
\par 首行缩进:2字符。
\par 行距:多倍行距1.25。

\section{封面}
\subsection{标题}
\par 中文标题:二号华文中宋和 Times New Roman 加粗,居中,左、右侧缩进均为4字符。
\par 外文标题:小二号 Times New Roman 加粗,居中,左、右侧缩进均为4字符。

\subsection{基本信息}
\par 基本信息是一个表格,左列为基本信息名称,右列为需要填写的基本信息。
\par 基本信息:四号华文中宋和 Times New Roman 加粗居中。

\section{承诺书}
\par 承诺书:三号华文中宋加粗,居中,段前4行,段后2行。
\par 承诺书文本:小四号宋体和 Times New Roman,首行缩进2字符,1.25倍行距。

\section{摘要}
\par 摘要:三号华文中宋加粗,居中,段前4行,段后2行。
\par 摘要文本:小四号宋体和 Times New Roman,首行缩进2字符,1.25倍行距。
\par 摘要文本结束后空一行。
\par 关键词:小四号宋体加粗顶格,××××:小四号宋体和 Times New Roman,各关键词之间2空格。

\section{ABSTRACT}
\par ABSTRACT:三号Times New Roman加粗,居中,段前4行,段后2行。
\par ABSTRACT文本:小四号Times New Roman,首行缩进2字符,1.25倍行距。
\par ABSTRACT文本结束后空一行。
\par KEY WORDS:小四号Times New Roman加粗顶格,××××:小四号Times New Roman,各关键词之间2空格。

\section{目录}
\par 目录:三号华文中宋加粗,居中,段前4行,段后2行。
\par 以下内容用小四号宋体和 Times New Roman,1.25倍行距:
\par 摘要:加粗,首行缩进2字符,段前0.5行;
\par ABSTRACT:加粗,首行缩进2字符,段前0.5行;
\par 第1章 ××××.........................1:加粗,首行缩进2字符,段前0.5行;
\par \quad 1.1 ××××............................1:加粗,首行缩进3字符;
\par \qquad 1.1.1 ××××.....................1:加粗,首行缩进4字符。

\section{正文}
\par 一级标题:三号华文中宋和 Times New Roman 加粗,居中,段前4行,段后2行。
\par 二级标题:四号宋体和 Times New Roman 加粗,左对齐顶格,段前0.5行,段后0行。
\par 三级标题:小四号宋体和 Times New Roman 加粗,左对齐顶格,段前0.5行,段后0行。
\par 正文文字:小四号宋体和 Times New Roman,首行缩进2字符,1.25倍行距。

\section{参考文献}
\par 参考文献:三号华文中宋加粗,居中,段前4行,段后2行。
\par 参考文献序号用方括号括起。
\par 参考文献序号和内容用五号宋体和 Times New Roman。

\section{致谢}
\par 致谢:三号华文中宋加粗,居中,段前4行,段后2行。
\par 致谢文本:小四号宋体和 Times New Roman,首行缩进2字符,1.25倍行距。
